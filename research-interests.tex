My primary research interests are in optimization of large design spaces, post-detonation nuclear forensics, prompt and delayed radiation effects, and radiation detector development. This research includes the interaction of radiation with matter, particularly focused on the characterization and improvement of nuclear radiation detectors and the effect radiation has on the electrical, optical, and mechanical characteristics of materials and devices. I have focused in three areas: the development neutron detectors for security applications, radiation effects on advanced electronic materials, and the effects of nuclear weapon outputs.
This research is multidisciplinary, involving nuclear physics, solid state physics, electrical and nuclear engineering, and other disciplines. The investigation of these materials had led to the use of many spectroscopic and characterization techniques to better understand the response of materials to radiation. It has also led to the development of many strong collaborations with universities including the University of Nebraska, Ohio State University, and the University of Michigan, as well as collaborations with government laboratories including Lawrence Livermore National Laboratory, the Air Force Research Laboratory, the Army Research Laboratory, and Berkeley National Laboratory. 
